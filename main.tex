%! suppress = FileNotFound
\documentclass[12pt]{amsart}
\usepackage{geometry} % see geometry.pdf on how to lay out the page. There's lots.
\geometry{a4paper} % or letter or a5paper or ... etc

\usepackage{url}
\renewcommand{\UrlFont}{\small\texttt{}}
% \geometry{landscape} % rotated page geometry

% See the ``Article customise'' template for come common customisations

% Potential sources:
% https://towardsdatascience.com/minis-minority-report-a-reality-pt-1-2-813104fcf1f
% https://futurism.com/the-byte/crime-prediction-ai-sucks-minority-report
% https://www.wired.com/story/a-british-ai-tool-to-predict-violent-crime-is-too-flawed-to-use/
% https://www.westmidlands-pcc.gov.uk/wp-content/uploads/2019/12/27112019-EC-Item-6-Minutes-and-Advice.pdf?x30523
% https://www.scmp.com/tech/science-research/article/3004167/minority-report-style-crime-prevention-artificial
% https://www.scmp.com/tech/science-research/article/3004167/minority-report-style-crime-prevention-artificial

\title{Are We Ready For An AI-based Minority Report?}
\author{Samuel Mason}

%%% BEGIN DOCUMENT
\begin{document}

    \maketitle
    \tableofcontents


    \section{Introduction}\label{sec:introduction}  % Get 300 words here

    In the book Minority Report, by Philip K Dick, the city police department uses a group of psychics to read
    the minds of the population.
    This allows the police to predict where, and when a crime will be committed by someone, which hence leads to a vast
    reduction in crime for the city, as the police could arrive before the crime, instead of after.
    Of course this raises some ethical issues, as well as some political issues, which will be covered below.
    \\
    While we don't currently have such psychics, we have an awful lot of data, and a vast amount of processing power, so
    what's to say we can't use that data, and AI, to do the jobs of the psychics.
    In this essay we will explore the viability of this, as well as the problems that could arise.

    \section{The problem}\label{sec:the-problem}  % Get 200 words here

    According to the ONS, the levels of crime have been increasing over the past 5 years ~\cite{ons-crime}, which can be
    put down to many reasons, such as a general increase in population.
    We have also seen a continuous decrease in the number of police officers, up until 2019 ~\cite{ho-pw}.
    This would very likely contribute to the increase in crime too, so clearly more has to be done to stop it.

    \section{The solution}\label{sec:the-solution}  % Get 300 words here

    One method to fix this could be to predict crime before it happens.
    While we don't have the Precogs from Minority Report, we do have very powerful artificial intelligence, which is
    already on its way to predicting some crimes.
    \\

    Take Fraud for instance, there are more and more ways to commit fraud, but it is also detectable, and preventable
    given the correct resources.
    If we collected all the financial data, and purchasing history of those who have been proven guilty for
    committing a certain fraud, such as Credit Card fraud, we could then monitor everyone's shopping trends and spending
    habits and compare them to the historical data.
    If this data starts to line up, then we could make the suggestion that said person is about to commit such fraud.
    \\

    Another crime that could be predicted is violent crime.
    While this is more difficult, as it requires monitoring physical environments, it is doable, especially with
    increased monitoring of the population.
    Almost everyone carries a smartphone with them, which can track location, and record sound, we also have extensive
    CCTV systems in built up places, both of these could be used to predict future crimes.
    This would be done by spotting behavioural changes in people, and comparing it to the behavioural changes in
    convicted criminals before they committed their respective crime.
    \\

    All of this analysis would require a huge amount of data, and processing power, so one method to gain this power
    is to use the personal devices of the people themselves to process the data.
    Many phones now have the power to run complex machine learning algorithms on board, and so, if each person's phone
    monitored them and their activities, it could predict the crime, and report it, before it even happened.
    This would be an already available resource to take the place of the Precogs in Minority Report, which don't yet
    exist.

    \section{Why should we do this?}\label{sec:why-should-we-do-this?}
    % Get 700 words here

There are some compelling reasons for the system mentioned above.
The main one being that the amount spent on human police resources would be able to be decreased, as fewer
police officers would be required to man the streets, or patrol.
They would only need to respond to a pre-crime, and hence, more money could be spent on community support, and
initiatives to help create jobs and opportunities in an area.
\\

However, this is may be undermined by existing crime gangs, who could become even more prominent, as they have
existing resources to figure out how to get around the prediction system, or may be able to "buy" their way into
the local community support in order to exert more power onto the local community.
This could be a dangerous effect, especially when it comes round to local youth populations, who may be drawn into
these gangs as a way to avoid the system, and go about their lives as they want, such as underage drinking, or
substance abuse.
\\

Furthermore, this system would create jobs all over the place, such as a new legal role, to try to defend those
convicted of a pre-crime, roles for managing the system, and keeping it working, and roles for implementing it in
the first place, like software developers, and technicians.
Over time, too, it will improve, as more and more data is collected, and more and more people become connected to
the system.
This is helpful as it wouldn't have a high increase in cost as the population grows, whereas a more traditional
method, of having police officers on patrol and such, would cost more, as more officers are needed to cope with
the rising population.
\\

The downside to this, however, is that the role of police officer may become unattractive, as the job becomes less
necessary, and fewer people take it up.
This could create a long term issue as existing police officers retire, and nobody is coming in to take
their places, and hence the threat of the system to people who may create a crime is reduced, and crime may then
increase.
To mitigate this, the role of police could be expanded into other areas instead, such as monitoring petty crimes
that are more difficult to predict than others, as they tend to be more spontaneous.
We have seen in 2020 that the number of police officers has already shot up (by 655\%), despite decreasing
for the past decade ~\cite{ho-pw}, this may have been down to the governments push for more public sector jobs,
nevertheless, this shows that currently the job is still attractive to people, and potentially adding a futuristic
crime prediction system could make the job more exciting to some, and easier for the rest.

    \section{Why shouldn't we do this?}\label{sec:why-shouldn't-we-do-this?}
    % Get 1000 words here

Despite its upsides, this system does come with some downsides.
Most notably, the fact that someone could just not be tracked if they turned their phone off, or if they switched
back to an old-fashioned device.
To mitigate this, more CCTV could be installed, as well as making it mandatory to carry a modern phone around with
you and keep it on at all times, which raises an even bigger concern, privacy.
If your phone is tracking everything you do then surely a concern would be where the data collected is stored, and
who is it shared with?
If the system were to be written into law, and use was mandatory, then there is a vast grey area where the
privacy policy that everyone usually agrees to would cover, so who the user's data is shared with is not up to the
government, and they could share it with whomever they want.
Of course, this would be discussed, and would likely include some link to GDPR regulation, and include a way for
each user to request what data they have stored on them, regardless of who they are, unless there is a risk to
national security if the user could see their data.
% More here
\\

To counter this, the argument could be made that, if you have nothing to hide, then you don't have to
worry.
If this mindset becomes the norm, then people will be happier to comply with the system, with the knowledge that
it keeps them safe, at the expense of their freedom.
This sort of echos what has happened recently with COVID-19, where people have given up their freedoms (freedom of
movement, and of choice) in order to better the health of the country, so the ability to get compliance from the
general population is proven if you frame it correctly.
We have also seen this sort of thing happen in China, with their social credit system, however, to implement such a
system here would likely require huge regime change at the top level, which would face a huge backlash, and
possibly a crime wave in protest.
So there must be a balance met if you want to get around this huge issue.
\\

Another major concern is that of corruption.
The data used could be meddled with by the providers in order to be more harsh on certain groups of people, such as
the middle to lower classes, those whom are not of the global elite.
This could then be used by these powerful individuals to keep the general population from rising up, and hence power
is kept by them.
Of course, this is a long shot, but if such a system was put into place, it would have to have protections against
this, which is a tricky area too, as what's to say if we can trust those 'protecting' us from it.

Furthermore, when this sort of system was trialed by West Midlands police, they removed location and ethnicity data
~\cite{wmp-trial}, which could arguably remove any form of economic and ethnic background bias, however, it was
found that there was a bias towards those of minority backgrounds.
There was also a study into a facial recognition system used on CCTV struggled more to identify those of colour,
than those not of colour ~\cite{ifsec-bias-article}.
There was found to be a 34\% error rate for identifying the gender of African american women, compared to the
0.8\% rate for white men.
While this could be down to insufficient, or erroneous training data, it shows that AI still has a way to go before
being good enough for CCTV video recognition.
\\

One the contrary, if the people in charge of the system were democratically voted in, and extensively independently
audited, then an argument could be made that the people in charge have the best intentions of the people who are
subject to the system.
Additionally, if all the data used for the system was publicly available, as well as all crime data for the
country, then the public would be able to expose any form of corruption.
Current national crime data would not extend far enough for this to work, as it doesn't cover people's full history,
which is required for an extensive, and accurate predictions.

Still, we do have a lot of data on people's habits, and spending, through online shopping receipts, and bank statements
which could become public in order for this to work, along with this, we could have messaging, and email, history.
If this was all made available to the system, then it would remove some elements of bias, but would also link back to
the aforementioned privacy concerns, and could make people more wary about what they talk about on their phones, as
it could end up being deemed dangerous.
In Scotland, for example, the new Hate Crime bill  ~\cite{hate-crime-bill}, would allow for private conversations that
were deemed to stir up hate, or be hateful in nature, to be prosecuted.
This sort of bill which covers private conversations could be abused, especially if our phones were to record
everything we say, and do, and could lead to people being prosecuted just for things they say.
This could be worrying for people, as they may not know what is deemed offensive, or in the case of a crime prediction
system, conspiracy to commit a crime, and hence they may feel that their free speech is threatened, and may revolt
against the system.
\\

Essential to the success of this system is of course, public trust, as the system would not even be able to take off
if most of the population were to reject it in its early stages.
The issue raised here could that of the quality of the Artificial Intelligence itself, as some countries are much
further ahead than ours in their AI technologies.
One of these countries is China ~\cite{ai-arms-race}, which has been under scrutiny recently due to their alleged
spying through their Huawei technology, which became popular in the Western World, up until it was famously banned in
the US, and its 5G network removed in the UK .

If China's world leading AI system was used in the UK for this crime prediction system, then public trust would
decrease, as people would believe a foreign state to be intervening in the affairs of the UK .
To get around this, the UK must first up its own game in the world of AI, as well as be clear on the ownership of the
companies that develop it, as well as the system, because, linking back to foreign intervention, a lot of big
infrastructure projects in the UK are owned by Chinese firms, such as the Hinckley Point nuclear project\cite{hinkley}.
This would make the system look even more dodgy, and the public would protest it, as well as those in government.

    \section{Conclusion}\label{sec:conclusion}  % Get 500 words here

    Sum it all up here, as if they didn't even read what was above ~\cite{mr-book}

    \bibliography{sources}{}
    \bibliographystyle{plain}

\end{document}
% Get 1000 words here

Despite its upsides, this system does come with some downsides.
Most notably, the fact that someone could just not be tracked if they turned their phone off, or if they switched
back to an old-fashioned device.
To mitigate this, more CCTV could be installed, as well as making it mandatory to carry a modern phone around with
you and keep it on at all times, which raises an even bigger concern, privacy.
If your phone is tracking everything you do then surely a concern would be where the data collected is stored, and
who is it shared with?
If the system were to be written into law, and use was mandatory, then there is a vast grey area where the
privacy policy that everyone usually agrees to would cover, so who the user's data is shared with is not up to the
government, and they could share it with whomever they want.
Of course, this would be discussed, and would likely include some link to GDPR regulation, and include a way for
each user to request what data they have stored on them, regardless of who they are, unless there is a risk to
national security if the user could see their data.
% More here
\\

To counter this, the argument could be made that, if you have nothing to hide, then you don't have to
worry.
If this mindset becomes the norm, then people will be happier to comply with the system, with the knowledge that
it keeps them safe, at the expense of their freedom.
This sort of echos what has happened recently with COVID-19, where people have given up their freedoms (freedom of
movement, and of choice) in order to better the health of the country, so the ability to get compliance from the
general population is proven if you frame it correctly.
We have also seen this sort of thing happen in China, with their social credit system, however, to implement such a
system here would likely require huge regime change at the top level, which would face a huge backlash, and
possibly a crime wave in protest.
So there must be a balance met if you want to get around this huge issue.
\\

Another major concern is that of corruption.
The data used could be meddled with by the providers in order to be more harsh on certain groups of people, such as
the middle to lower classes, those whom are not of the global elite.
This could then be used by these powerful individuals to keep the general population from rising up, and hence power
is kept by them.
Of course, this is a long shot, but if such a system was put into place, it would have to have protections against
this, which is a tricky area too, as what's to say if we can trust those 'protecting' us from it.

Furthermore, when this sort of system was trialed by West Midlands police, they removed location and ethnicity data
~\cite{wmp-trial}, which could arguably remove any form of economic and ethnic background bias, however, it was
found that there was a bias towards those of minority backgrounds.
There was also a study into a facial recognition system used on CCTV struggled more to identify those of colour,
than those not of colour ~\cite{ifsec-bias-article}.
There was found to be a 34\% error rate for identifying the gender of African american women, compared to the
0.8\% rate for white men.
While this could be down to insufficient, or erroneous training data, it shows that AI still has a way to go before
being good enough for CCTV video recognition.
\\

One the contrary, if the people in charge of the system were democratically voted in, and extensively independently
audited, then an argument could be made that the people in charge have the best intentions of the people who are
subject to the system.
Additionally, if all the data used for the system was publicly available, as well as all crime data for the
country, then the public would be able to expose any form of corruption.
Current national crime data would not extend far enough for this to work, as it doesn't cover people's full history,
which is required for an extensive, and accurate predictions.

Still, we do have a lot of data on people's habits, and spending, through online shopping receipts, and bank statements
which could become public in order for this to work, along with this, we could have messaging, and email, history.
If this was all made available to the system, then it would remove some elements of bias, but would also link back to
the aforementioned privacy concerns, and could make people more wary about what they talk about on their phones, as
it could end up being deemed dangerous.
In Scotland, for example, the new Hate Crime bill  ~\cite{hate-crime-bill}, would allow for private conversations that
were deemed to stir up hate, or be hateful in nature, to be prosecuted.
This sort of bill which covers private conversations could be abused, especially if our phones were to record
everything we say, and do, and could lead to people being prosecuted just for things they say.
This could be worrying for people, as they may not know what is deemed offensive, or in the case of a crime prediction
system, conspiracy to commit a crime, and hence they may feel that their free speech is threatened, and may revolt
against the system.
% Get 300 words here

One method to fix this could be to predict crime before it happens.
While we don't have the Precogs from Minority Report, we do have very powerful artificial intelligence, which is
already on its way to predicting some crimes.
\\

One crime that could be predicted is violent crime.
While this is more difficult, as it requires monitoring physical environments, it is doable, especially with
increased monitoring of the population.
Almost everyone carries a smartphone with them, which can track location, and record sound, we also have extensive
CCTV systems in built up places, both of these could be used to predict future crimes.
This would be done by spotting behavioural changes in people, and comparing it to the behavioural changes in
convicted criminals before they committed their respective crime.

Currently, there are some trends that can be seen from the criminal record of people.
Such as the re-offending rates of those out of prison ~\cite{dci-walker}: If we took the time that they committed
their first crime, and then the time they committed their second one after being released, then we could somewhat
predict the rough date of the next offence.
Of course this could be quite inaccurate, but it can become more accurate when adding other factors, such as
financial history, which could provide a motive (perhaps for robbery).
\\

Another crime could be fraud, there are more and more ways to commit fraud, but it is also detectable, and preventable
given the correct resources.
If we collected all the financial data, and purchasing history of those who have been proven guilty for
committing a certain fraud, such as Credit Card fraud, we could then monitor everyone's shopping trends and spending
habits and compare them to the historical data.
If this data starts to line up, then we could make the suggestion that said person is about to commit such fraud.

One example that links to this is that loan, and credit cards already use ur spending habits to detect if you are
struggling financially, say for example, if you were shopping at Waitrose at the start of the month, then started
shopping at Aldi instead, then it would suggest that you are struggling financially, and may end up defaulting on
your loan.
This is can be used when giving out some credit scores, but these factors could also be used to predict some forms
of fraud, as well as other crimes, like burglary, which when linked with the re-offending rate mentioned above.
\\

All of this analysis would require a huge amount of data, and processing power, so one method to gain this power
is to use the personal devices of the people themselves to process the data.
Many phones now have the power to run complex machine learning algorithms on board, and so, if each person's phone
monitored them and their activities, it could predict the crime, and report it, before it even happened.
This would be an already available resource to take the place of the Precogs in Minority Report, which don't yet
exist.
% Get 700 words here

There are some compelling reasons for the system mentioned above.
The main one being that the amount spent on human police resources would be able to be decreased, as fewer
police officers would be required to man the streets, or patrol.
They would only need to respond to a pre-crime, and hence, more money could be spent on community support, and
initiatives to help create jobs and opportunities in an area.
\\

However, this is may be undermined by existing crime gangs, who could become even more prominent, as they have
existing resources to figure out how to get around the prediction system, or may be able to "buy" their way into
the local community support in order to exert more power onto the local community.
This could be a dangerous effect, especially when it comes round to local youth populations, who may be drawn into
these gangs as a way to avoid the system, and go about their lives as they want, such as underage drinking, or
substance abuse.
\\

Furthermore, this system would create jobs all over the place, such as a new legal role, to try to defend those
convicted of a pre-crime, roles for managing the system, and keeping it working, and roles for implementing it in
the first place, like software developers, and technicians.
Over time, too, it will improve, as more and more data is collected, and more and more people become connected to
the system.
This is helpful as it wouldn't have a high increase in cost as the population grows, whereas a more traditional
method, of having police officers on patrol and such, would cost more, as more officers are needed to cope with
the rising population.
\\

The downside to this, however, is that the role of police officer may become unattractive, as the job becomes less
necessary, and fewer people take it up.
This could create a long term issue as existing police officers retire, and nobody is coming in to take
their places, and hence the threat of the system to people who may create a crime is reduced, and crime may then
increase.
To mitigate this, the role of police could be expanded into other areas instead, such as monitoring petty crimes
that are more difficult to predict than others, as they tend to be more spontaneous.
We have seen in 2020 that the number of police officers has already shot up (by 655\%), despite decreasing
for the past decade ~\cite{ho-pw}, this may have been down to the governments push for more public sector jobs,
nevertheless, this shows that currently the job is still attractive to people, and potentially adding a futuristic
crime prediction system could make the job more exciting to some, and easier for the rest.